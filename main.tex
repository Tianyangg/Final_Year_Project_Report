\documentclass{article}
\usepackage[utf8]{inputenc}
\usepackage{geometry}
\usepackage{amsmath}
\geometry{a4paper,left=2.5cm,right=2.5cm,top=2cm,bottom=1.5cm}

\title{Final Year Project Report}
\author{txs799 }
\date{March 2019}

\begin{document}

\maketitle

\section{Introduction}
\subsection{Bayesian networks}
\subsection{CNF}
\subsection{Model Counting and Weighted Model Counting}

\section{Why can we encode BN as CNF}

\section{The evolution of encodings}
\subsection{Full Encoding}
\subsubsection{Generating CNFs}
Given a Bayesian Network, two types of variables are generated.\\
Define the variables generated from Node \textbf{\textit{X}} as Indicator Variables.
For a node \textbf{\textit{X}} in Bayesian Network \textbf{N}, let $\lambda_x$ define the indicator variable. \\
Define the variables generated form Node \textbf{\textit{X}} and its parents \textbf{Y} = {$Y_{1}$, ... $Y_{n}$} as Parameter Variables.
For a node \textbf{\textit{X}} in Bayesian Network \textbf{N}, let $\theta_{X|Y}$ denotes the parameter variable.\\
\textbf{Obtaining indicator clauses \textsc{I}:}\\
For each node \textit{X} in a Bayesian Network with probability \{$x_{1}$,... ,$x_{n}$\} $\in$ \textit{X}, the following clauses are generated:
\begin{equation}\label{fullenc_ic1}
    \lambda_{x_{1}} \vee ... \vee \lambda_{x_{n}}
\end{equation}

\begin{equation}\label{eq:fullenc_ic2}
    \neg\lambda_{x_{i}} \vee \neg\lambda_{x_{j}}, \;\;\; \mbox{for each i $\neq$ j}
\end{equation}
According to the commutation of logic OR, $R \vee Q$ $\Longleftrightarrow$ $Q \vee R$, to avoid redundant clauses, \ref{eq:fullenc_ic2} can be simplified as :
\begin{equation}\label{fullenc_ic3}
    \neg\lambda_{x_{i}} \vee \neg\lambda_{x_{j}}, \;\;\; \mbox{for each i $<$ j}
\end{equation}
\textbf{Obtaining parameter clauses \textsc{P}:}\\
For each node \textbf{\textit{X}} in a Bayesian Network and its parents \textbf{\textit{Y}}, the following clauses are generated:
\begin{equation}\label{fullenc_pc1}
    \lambda_{x_{i}} \wedge \lambda_{y_{1}} \wedge... \wedge \lambda_{y_{m}} \leftrightarrow \theta_{x_{i}|y_{1}..y{m}}
\end{equation}
The following equation is the equivalent of \ref{fullenc_pc1} written in the way that meet the requirement of CNF.
\begin{equation}\label{fullenc_IP}
    \neg\lambda_{x_{i}} \vee \neg\lambda_{y_{1}} \vee... \vee \neg\lambda_{y_{m}} \vee \theta_{x_{i}|y_{1}..y_{m}}
\end{equation}
\begin{equation}\label{fullenc_PI}
    \neg\theta_{x_{i}|y_{1}..y_{m}} \vee \lambda_{x_{i}},\\ \;\;
    \neg\theta_{x_{i}|y_{1}..y_{m}} \vee \lambda_{y_{j}} \;\; \mbox{ j = 1, ..., m}
\end{equation}
%% Here should be the example with BN asia
\subsubsection{Assigning Weights}
% how weights are assigned
\subsection{An improvement of Full Encoding}
\subsubsection{A node with cardinality = 2}
Now consider the node with cardinality 2. Take an Example in asia network, node \textit{smoker} has two possibilty \textit{Yes} and \textit{No}, so according to \ref{fullenc_ic1}, we have $\lambda_{smoker_{yes}} \vee \lambda_{smoker_{no}}$ and this will always be 1, and the same apply for \ref{fullenc_ic3}: $\neg\lambda_{smoker_{yes}} \vee \neg\lambda_{smoker_{no}}$\\
\textit{\textbf{Simplification step 1:}} If the cardinality of a node in a Bayesian Network is 2, ommit the clause in \ref{fullenc_ic1} and \ref{fullenc_ic3} in \textit{3.1 Full Encoding}.

\subsubsection{Parameter variable = 1 or parameter variable = 0}

\subsection{Improved Encoding}

\subsection{Group Encoding}
\subsubsection{The idea of simplification by preprocessing the CNF}
\subsection{The idea of QM Algorithm}
\subsection{An extension of How the multievel simplification }


\end{document}
